\chapter{Capítulo 1}

No desenvolvimento é onde se: expõe, explica, demonstra, fundamenta, prova. É a comunicação dos trabalhos desenvolvidos e dos resultados obtidos. Geralmente esta seção é dividida em subcapítulos que devem começar com uma pequena introdução e terminar com uma conclusão, onde aspectos relevantes são convenientemente ressaltados.

\section{No capítulo 1}

Introdução à problemática global dentro da qual o problema específico que o trabalho de estágio ou PFC tratará se enquadra; motivação e justificativa para se resolver o problema (ou seja, o que está mal hoje e porque algo como o que se pretende fazer resolverá, mesmo que parcialmente); escopo do projeto e objetivos do estágio/PFC; parágrafo final que descreve a estrutura e lógica geral do documento.

Ainda, sobre a metodologia usada. Não confundir metodologia com a lista das fases do cronograma. Aqui se refere em \textbf{como} (técnicas e procedimentos) cada uma das fases foi planejada e depois executada. Aqui, e eventualmente em algumas partes do documento, explicitar também o que o \textbf{aluno} quem tenha realmente feito, e não apenas o que uma equipe fez.

\textbf{Exemplos de citação:}

\cite{van86}

\citeonline{van86}

\textbf{Exemplo de imagem} - \autoref{fig_exemplo}:

\begin{figure}[htb]
	\caption{\label{fig_exemplo}Título da figura}
	\begin{center}
	    \includegraphics[scale=0.2]{imagens/Example.png}
	\end{center}
	\legend{Fonte: Arquivo pessoal.}
\end{figure}