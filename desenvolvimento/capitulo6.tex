\chapter{Capítulo 6}

Análise  dos  resultados:  análise  crítica  e  profunda  sobre  o  que fizeram,  os  resultados atingidos,  os  prós  e  contras,  impactos  da sua  utilização  na  empresa,  como  "provam"  que  isso resolveu  (ou  melhorou)  o  problema  inicialmente  identificado,  os  indicadores usados  para avaliação (e o porquê deles e não outros), os ganhos quanti ou qualitativos obtidos, etc..


Onde se apresentam e são discutidos os resultados (simulados, piloto ou reais) obtidos no decorrer da pesquisa/desenvolvimento. 
Nesta discussão é importante dar meios para que o leitor entenda os resultados obtidos e como foram obtidos.
Deve-se fazer uma análise dos resultados: análise crítica profunda sobre o que foi feito, os resultados atingidos, os prós e contras, impactos, como isso resolveu o problema, os ganhos obtidos, etc.