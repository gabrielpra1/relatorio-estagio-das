Onde, de maneira sucinta, apresenta-se:

\begin{itemize}
\item 	definição do assunto, o problema a ser abordado no trabalho;
\item 	localização do assunto no espaço, no tempo e no contexto do Curso;
\item 	importância do assunto e justificativa da escolha;
\item 	objeto geral e objetivos específicos;
\item 	hipóteses levantadas ou argumentação principal;
\item 	descrição da metodologia empregada no trabalho (bibliografia e/ou estudo de campo e/ou de laboratório);
\item 	as alternativas de solução do Problema, decorrentes de estudo bibliográfico que deverá ficar evidenciado mediante referências;
\item 	apresentação do plano adotado para o desenvolvimento do assunto, isto é, a descrição do que será tratado em cada um dos capítulos subseqüentes.
\end{itemize}

    
\textbf{Importante:} Ao longo de todo o texto da monografia, quando pertinente, deve-se procurar contextualizar e explicitar em quais atividades vocês (e não as outras pessoas da equipe / empresa) atuaram / estiveram envolvidos, e o que vocês efetivamente fizeram dentro do todo apresentado.