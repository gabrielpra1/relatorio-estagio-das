% ------------------------------------------------------------------------
% ------------------------------------------------------------------------
% Modelo de relatório de estágio DAS - UFSC.
% Esse modelo usa abnTeX2 (https://www.abntex.net.br/)
% ------------------------------------------------------------------------
% ------------------------------------------------------------------------

\documentclass[
	% -- opções da classe memoir --
	12pt,				% tamanho da fonte
	openright,			% capítulos começam em pág ímpar (insere página vazia caso preciso)
	oneside,			% Para impressão simples. Para impressão frente e verso, use twoside
	a4paper,			% tamanho do papel. 
	% -- opções do pacote babel --
	english,			% idioma adicional para hifenização
	brazil				% o último idioma é o principal do documento
	]{abntex2}

% ---
% Pacotes básicos 
% ---
\usepackage{lmodern}			% Usa a fonte Latin Modern			
\usepackage[T1]{fontenc}		% Selecao de codigos de fonte.
\usepackage[utf8]{inputenc}		% Codificacao do documento (conversão automática dos acentos)
\usepackage{lastpage}			% Usado pela Ficha catalográfica
\usepackage{indentfirst}		% Indenta o primeiro parágrafo de cada seção.
\usepackage{color}				% Controle das cores
\usepackage{graphicx}			% Inclusão de gráficos
\usepackage{microtype} 			% para melhorias de justificação
% ---
		
% ---
% Pacotes de citações
% ---
\usepackage[brazilian,hyperpageref]{backref}	 % Paginas com as citações na bibl
\usepackage[alf]{abntex2cite}	% Citações padrão ABNT

% --- 
% CONFIGURAÇÕES DE PACOTES
% --- 

% ---
% Configurações do pacote backref
% Usado sem a opção hyperpageref de backref
\renewcommand{\backrefpagesname}{Citado na(s) página(s):~}
% Texto padrão antes do número das páginas
\renewcommand{\backref}{}
% Define os textos da citação
\renewcommand*{\backrefalt}[4]{
	\ifcase #1 %
		Nenhuma citação no texto.%
	\or
		Citado na página #2.%
	\else
		Citado #1 vezes nas páginas #2.%
	\fi}%
% ---

% ---
% Informações de dados para CAPA e FOLHA DE ROSTO
% ---
\titulo{Titulo: subtítulo}
\autor{Nome do aluno}
\local{Florianópolis}
\data{2018}
\orientador{Nome do orientador}
\coorientador{Nome do coorientador} % Deixar vazio caso não haja
\instituicao{%
  UNIVERSIDADE FEDERAL DE SANTA CATARINA
  \par
  CENTRO TECNOLÓGICO
  \par
  DEPARTAMENTO DE AUTOMAÇÃO E SISTEMAS}
\tipotrabalho{Relatório de Estágio}

\preambulo{Relatório submetido à Universidade Federal de Santa Catarina como requisito para a aprovação na disciplina \textbf{DAS 5501 - Estágio em Controle e Automação Industrial} do curso de Graduação em Engenharia de Controle e Automação.}
% ---

% ---
% Configurações de aparência do PDF final

% alterando o aspecto da cor azul
\definecolor{blue}{RGB}{41,5,195}

% informações do PDF
\makeatletter
\hypersetup{
     	%pagebackref=true,
		pdftitle={\@title}, 
		pdfauthor={\@author},
    	pdfsubject={\imprimirpreambulo},
	    pdfcreator={LaTeX with abnTeX2},
		pdfkeywords={abnt}{latex}{abntex}{abntex2}{trabalho acadêmico}, 
		colorlinks=true,       		% false: boxed links; true: colored links
    	linkcolor=black,          	% color of internal links
    	citecolor=blue,        		% color of links to bibliography
    	filecolor=magenta,      		% color of file links
		urlcolor=blue,
		bookmarksdepth=4
}
\makeatother
% --- 

% --- 
% Espaçamentos entre linhas e parágrafos 
% --- 

% O tamanho do parágrafo é dado por:
\setlength{\parindent}{1.3cm}

% Controle do espaçamento entre um parágrafo e outro:
\setlength{\parskip}{0.2cm}  % tente também \onelineskip

% ---
% compila o indice
% ---
\makeindex
% ---

% ----
% Início do documento
% ----
\begin{document}

% Retira espaço extra obsoleto entre as frases.
\frenchspacing 

% ----------------------------------------------------------
% ELEMENTOS PRÉ-TEXTUAIS
% ----------------------------------------------------------
% \pretextual

% ---
% Capa
% ---

\renewcommand{\imprimircapa}{%
  \begin{capa}%
    \center
    \ABNTEXchapterfont\Large\textbf\imprimirinstituicao
    
    \vspace*{1cm}
    
    {\ABNTEXchapterfont\large\textbf\imprimirautor}
    
    \vfill
    \begin{center}
    \ABNTEXchapterfont\bfseries\ Relatório de Estágio \\
    \ABNTEXchapterfont\bfseries\LARGE\imprimirtitulo
    \end{center}
    \vfill
    
    \large\imprimirlocal
    
    \large\imprimirdata
    
    \vspace*{1cm}
  \end{capa}
}

\imprimircapa
% ---

% ---
% Folha de rosto
% (o * indica que haverá a ficha bibliográfica)
% ---
\makeatletter
\renewcommand{\folhaderostocontent}{
  \begin{center}
  
    {\ABNTEXchapterfont\large\textbf\imprimirautor}
    
    \vspace*{\fill}\vspace*{\fill}
    
    \begin{center}
    	\ABNTEXchapterfont\bfseries\Large\imprimirtitulo
    \end{center}
    
    \vspace*{\fill}
    
    \abntex@ifnotempty{\imprimirpreambulo}{%
      \hspace{.45\textwidth}
      \begin{minipage}{.5\textwidth}
      \SingleSpacing
      \imprimirpreambulo
      \par
      {\imprimirorientadorRotulo~\imprimirorientador\par}
      \abntex@ifnotempty{\imprimircoorientador}{%
    	{\imprimircoorientadorRotulo~\imprimircoorientador}%
      }%
      \end{minipage}%
      \vspace*{\fill}
    }%
    
    \vspace*{\fill}
    
    {\large\imprimirlocal}
    
    \par
    
    {\large\imprimirdata}
    
    \vspace*{1cm}
  \end{center}
}
\makeatother

\imprimirfolhaderosto
% ---

% ---
% Inserir folha de aprovação
% ---

% Isto é um exemplo de Folha de aprovação, elemento obrigatório da NBR
% 14724/2011 (seção 4.2.1.3). Você pode utilizar este modelo até a aprovação
% do trabalho. Após isso, substitua todo o conteúdo deste arquivo por uma
% imagem da página assinada pela banca com o comando abaixo:
%
% \includepdf{folhadeaprovacao_final.pdf}
%
\makeatletter
\begin{folhadeaprovacao}

  \begin{center}
    {\ABNTEXchapterfont\large\imprimirautor}

    \vspace*{\fill}\vspace*{\fill}
    \begin{center}
      \ABNTEXchapterfont\bfseries\Large\imprimirtitulo
    \end{center}
    \vspace*{\fill}
    
    Este relatório de estágio foi julgado no contexto da disciplina DAS5501: Estágio em Controle e Automação Industrial e \textbf{APROVADO} na sua forma final pelo Curso de Engenharia de Controle e Automação.
    
    \vspace*{\fill}
    
    Florianópolis, 22 de junho de 2018
  \end{center}
  
   \assinatura{\textbf{\imprimirorientador} \\ Orientador \\ Universidade Federal de Santa Catarina}
   
   \assinatura{\textbf{Nome do Supervisor} \\ Supervisor na Empresa \\ EMPRESA}
      
   \abntex@ifnotempty{\imprimircoorientador}{%
     \assinatura{\textbf{\imprimircoorientador} \\ Co-orientador \\ Universidade Federal de Santa Catarina}
   }%
\makeatother
   
        
\end{folhadeaprovacao}
% ---

% ---
% Dedicatória
% ---
\begin{dedicatoria}
   \vspace*{\fill}
   \centering
   \noindent
   \textit{Dedicatória: OPCIONAL} \vspace*{\fill}
\end{dedicatoria}
% ---

% ---
% Agradecimentos
% ---
\begin{agradecimentos}
(OPCIONAL): Essa seção não possui uma estrutura pré-definida. O mais usual é separar em parágrafos. Mencionar apenas pessoas, instituições e outros que tenham tido importância para a realização do trabalho.

\end{agradecimentos}
% ---

% ---
% Epígrafe
% ---
\begin{epigrafe}
    \vspace*{\fill}
	\begin{flushright}
		\textit{``Epígrafe (Opcional – NBR10520). Elemento opcional, no qual o autor apresenta uma citação, seguida de indicação de autoria, relacionada à matéria tratada no corpo do trabalho.''}
	\end{flushright}
\end{epigrafe}
% ---

% ---
% RESUMOS
% ---

% resumo em português
\setlength{\absparsep}{18pt} % ajusta o espaçamento dos parágrafos do resumo
\begin{resumo}
	 Apresenta as informações principais do documento (Descrição geral da empresa (natureza, mercado, processos, etc.), problema-foco atacado no PFC, o que foi feito, principais resultados atingidos, etc.). Deve conter entre 100 e 500 palavras (NBR 6028/2003) em um único parágrafo. Se o documento for escrito em outra língua que não o Português, então é necessário fazer um Resumo \textbf{Estendido} em Português, ao invés deste resumo enxuto.

\vspace{\onelineskip}

\noindent 

\textbf{Palavras-chave:} No mínimo 3 (três) e separadas por ponto (.) 
\end{resumo}

% resumo em inglês
\begin{resumo}[Abstract]
 \begin{otherlanguage*}{english}
	Resumo em língua inglesa. Mesma formatação do Resumo.

\vspace{\onelineskip}

\noindent 
\textbf{Key-words}: No mínimo 3 (três) e separadas por ponto (.) 
 \end{otherlanguage*}
\end{resumo}

% Inserir listas. Todas essas listas são opcionais.
% ---
% inserir lista de ilustrações
% ---
\pdfbookmark[0]{\listfigurename}{lof}
\listoffigures*
OPCIONAL
\cleardoublepage
% ---

% ---
% inserir lista de tabelas
% ---
\pdfbookmark[0]{\listtablename}{lot}
\listoftables*
OPCIONAL
\cleardoublepage
% ---

% ---
% inserir lista de abreviaturas e siglas
% ---
\begin{siglas}
  \item[ABNT] OPCIONAL
\end{siglas}
% ---

% ---
% inserir lista de símbolos
% ---
\begin{simbolos}
  \item[$ \Gamma $] OPCIONAL
\end{simbolos}
% ---


% ---
% inserir o sumario
% ---
\pdfbookmark[0]{\contentsname}{toc}
\tableofcontents*
\cleardoublepage
% ---



% ----------------------------------------------------------
% ELEMENTOS TEXTUAIS
% ----------------------------------------------------------
\textual

% ----------------------------------------------------------
% Introdução (exemplo de capítulo sem numeração, mas presente no Sumário)
% ----------------------------------------------------------
\chapter*[Introdução]{Introdução}
\addcontentsline{toc}{chapter}{Introdução}
% ----------------------------------------------------------

Onde, de maneira sucinta, apresenta-se:

\begin{itemize}
\item 	definição do assunto, o problema a ser abordado no trabalho;
\item 	localização do assunto no espaço, no tempo e no contexto do Curso;
\item 	importância do assunto e justificativa da escolha;
\item 	objeto geral e objetivos específicos;
\item 	hipóteses levantadas ou argumentação principal;
\item 	descrição da metodologia empregada no trabalho (bibliografia e/ou estudo de campo e/ou de laboratório);
\item 	as alternativas de solução do Problema, decorrentes de estudo bibliográfico que deverá ficar evidenciado mediante referências;
\item 	apresentação do plano adotado para o desenvolvimento do assunto, isto é, a descrição do que será tratado em cada um dos capítulos subseqüentes.
\end{itemize}

    
\textbf{Importante:} Ao longo de todo o texto da monografia, quando pertinente, deve-se procurar contextualizar e explicitar em quais atividades vocês (e não as outras pessoas da equipe / empresa) atuaram / estiveram envolvidos, e o que vocês efetivamente fizeram dentro do todo apresentado.


% Desenvolvimento
\chapter{Capítulo 1}

No desenvolvimento é onde se: expõe, explica, demonstra, fundamenta, prova. É a comunicação dos trabalhos desenvolvidos e dos resultados obtidos. Geralmente esta seção é dividida em subcapítulos que devem começar com uma pequena introdução e terminar com uma conclusão, onde aspectos relevantes são convenientemente ressaltados.

\section{No capítulo 1}

Introdução à problemática global dentro da qual o problema específico que o trabalho de estágio ou PFC tratará se enquadra; motivação e justificativa para se resolver o problema (ou seja, o que está mal hoje e porque algo como o que se pretende fazer resolverá, mesmo que parcialmente); escopo do projeto e objetivos do estágio/PFC; parágrafo final que descreve a estrutura e lógica geral do documento.

Ainda, sobre a metodologia usada. Não confundir metodologia com a lista das fases do cronograma. Aqui se refere em \textbf{como} (técnicas e procedimentos) cada uma das fases foi planejada e depois executada. Aqui, e eventualmente em algumas partes do documento, explicitar também o que o \textbf{aluno} quem tenha realmente feito, e não apenas o que uma equipe fez.

\textbf{Exemplos de citação:}

\cite{van86}

\citeonline{van86}

\textbf{Exemplo de imagem} - \autoref{fig_exemplo}:

\begin{figure}[htb]
	\caption{\label{fig_exemplo}Título da figura}
	\begin{center}
	    \includegraphics[scale=0.2]{imagens/Example.png}
	\end{center}
	\legend{Fonte: Arquivo pessoal.}
\end{figure}
\chapter{Capítulo 2}

Descrição da empresa, processos, layouts, problemas, indicadores existentes, etc.

No Capítulo 2, deve-se fazer uma descrição da empresa, processos, layouts, problemas, etc., ou seja, mais detalhadamente motivar e enquadrar o problema dentro do que proporão (a ser descrito no próximo capítulo).
\chapter{Capítulo 3}

Aspectos conceituais: "teorias", modelos, etc. que usaram na solução do problema e suas justificativas  para  o  seu  uso  (e  não  o  de outras,  que  teoricamente  teriam  sido  estudadas  e comparadas, para então se decidir pelo seu uso).

Onde são formalizados o problema e as técnicas para a sua solução. Devem-se apresentar os aspectos conceituais: teorias gerais que usaram na solução do problema e suas breves análises (uma vez que algumas delas ou todas as apresentadas aqui serão posteriormente mencionadas nos próximos capítulos, quando se discorrerá sobre o que foi efetivamente realizado).
Dependendo do caso pode ser necessário mais de um capítulo para tal. 
\chapter{Capítulo 4}

Requisitos gerais,  funcionais  e  não  funcionais, a  serem  cumpridos com  o  que  se  propõe desenvolver,  descrição  *conceitual*  do  que fizeram,    do  novo  jeito  de  funcionar,  fluxos  de informação e materiais, diagramas, etc., deixando claro em que medida essas coisas resolvem o problema  que  se  deseja,  as  decisões  que  foram  toma das  e  o  porquê  /  embasamento  dessas decisões, e alinhamento disso àqueles requisitos.

Deve-se fazer uma descrição “conceitual” do que fizeram, da nova maneira de funcionar, fluxos de informação e materiais, diagramas, modelos, arquiteturas, etc., deixando claro em que medida o que e forma como projetaram / idealizaram atacam o problema identificado na Introdução. 
Aqui também deve ser colocada a Metodologia usada (não confundir isso com a lista das fases do cronograma), no sentido de “como” (o raciocínio lógico, os modelos dos quais se buscou o embasamento teórico, etc.) e com base no que propuseram a proposta.
Pode, também, ser dividido em mais de um capítulo se for o caso.

\chapter{Capítulo 5}

Descrição  do  "projeto"  e/ou  software  que  se  fez,  diagramas, Casos  de  Uso,  interfaces gráficas, ambientes de desenvolvimento e TIs usadas (e porque as selecionaram dentro de várias possibilidades), etc.

Onde se aborda a implementação/implantação da solução escolhida. 
Deve-se fazer uma descrição do "projeto" e/ou software que se fez, diagramas, Casos de Uso, interfaces gráficas, ambientes de desenvolvimento (e porque os escolheram), TIs usadas (e porque as selecionaram), etc.
Aqui também deve ser colocada a Metodologia usada (não confundir isso com a lista das fases do cronograma), no sentido de “como” (técnicas e procedimentos) cada uma das fases foi planejada e executada. Por exemplo, com que técnica ou procedimento se testou, avaliou ou validou a solução, se verificou se ela realmente atendeu aos anseios dos usuários, se ela atendeu a todos os requisitos especificados, etc.
\chapter{Capítulo 6}

Análise  dos  resultados:  análise  crítica  e  profunda  sobre  o  que fizeram,  os  resultados atingidos,  os  prós  e  contras,  impactos  da sua  utilização  na  empresa,  como  "provam"  que  isso resolveu  (ou  melhorou)  o  problema  inicialmente  identificado,  os  indicadores usados  para avaliação (e o porquê deles e não outros), os ganhos quanti ou qualitativos obtidos, etc..


Onde se apresentam e são discutidos os resultados (simulados, piloto ou reais) obtidos no decorrer da pesquisa/desenvolvimento. 
Nesta discussão é importante dar meios para que o leitor entenda os resultados obtidos e como foram obtidos.
Deve-se fazer uma análise dos resultados: análise crítica profunda sobre o que foi feito, os resultados atingidos, os prós e contras, impactos, como isso resolveu o problema, os ganhos obtidos, etc.
\chapter{Capítulo 7}

Conclusões gerais: "resumão" do que foi feito e dos resultados globais, limitações do que foi desenvolvido e pressupostos assumidos, e sugestão para trabalhos futuros de continuação.

Síntese pessoal, objetiva, sucinta e interpretada dos resultados do trabalho.
Grosso modo, deve-se apresentar um resumo do que foi feito, dos resultados globais (frente aos objetivos inicialmente traçados). Exemplos:

\begin{itemize}
\item o método deu certo? funcionou? deu o resultado esperado? Foi melhor que o método anterior?
\item impactos organizacionais, tecnológicos, financeiros, éticos, ecológicos, etc., tidos (ou potencialmente a ter) com a introdução do que foi proposto.
\end{itemize}


De forma complementar, se pertinente, sugestões para trabalhos futuros de continuação.


% ----------------------------------------------------------
% Finaliza a parte no bookmark do PDF
% para que se inicie o bookmark na raiz
% e adiciona espaço de parte no Sumário
% ----------------------------------------------------------
\phantompart

% ----------------------------------------------------------
% ELEMENTOS PÓS-TEXTUAIS
% ----------------------------------------------------------
\postextual
% ----------------------------------------------------------

% ----------------------------------------------------------
% Referências bibliográficas
% ----------------------------------------------------------
\bibliography{bibliografia}

% ----------------------------------------------------------
% Apêndices
% ----------------------------------------------------------

% ---
% Inicia os apêndices
% ---
\begin{apendicesenv}

\chapter{Título Opcional}
Textos elaborados pelo próprio autor, a fim de complementar a sua argumentação

\end{apendicesenv}
% ---


% ----------------------------------------------------------
% Anexos
% ----------------------------------------------------------

% ---
% Inicia os anexos
% ---
\begin{anexosenv}

% ---
\chapter{Título Opcional.}
% ---
Documentos não elaborados pelo autor, utilizados de fundamentação (mapas, leis, estatutos).

\end{anexosenv}

%---------------------------------------------------------------------
% INDICE REMISSIVO
%---------------------------------------------------------------------
\phantompart
\printindex
%---------------------------------------------------------------------

\end{document}
